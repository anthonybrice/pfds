% arara: latexmk: { engine: pdflatex, options: ['-shell-escape'] }
% arara: latexmk: { engine: pdflatex, clean: partial }

\documentclass[nobib, letterpaper]{tufte-handout}

\usepackage{xcolor}
\definecolor{LightGray}{gray}{0.9}
\usepackage{minted}
\usepackage{ccfonts}
\usepackage{euler}
\renewcommand*\ttdefault{cmtt}
\usepackage{microtype}

\usepackage[T1]{fontenc}

\usepackage{units}

\date{} % clear date

\usepackage{tikz}
\usetikzlibrary{calc,shapes.multipart,chains,arrows.meta}

\title{Purely Functional Data Structures}
\author{Chapter 2}


\geometry{
  left=.75in, % left margin
  textwidth=35pc, % main text block
  marginparsep=2pc, % gutter between main text block and margin notes
  marginparwidth=12pc % width of margin notes
}

\begin{document}

\maketitle

\section{Exercise 1}

We define our function as a right fold that combines the accumulator, the accumulator's head element, and the present element of the fold.

\inputminted
  [bgcolor=LightGray,firstline=11, lastline=14]
  {haskell}
  {src/Lib.hs}

Consider \mintinline[bgcolor=LightGray]{haskell}{suffixes input}. To generate its output, we merely traverse the list once attaching an additional pointer to each node. Thus the procedure can be performed in $O(n)$ time and requires only an extra $n=O(n)$ space.

\begin{figure*}
  \begin{tikzpicture}[
    list/.style={
        rectangle split,
        rectangle split parts=2,
        draw,
        rectangle split horizontal
      },
      >=stealth,
      start chain=xs,
      node distance=1em
    ]

    \node (E) {input};
    \node[right=of E, list,on chain=xs] (A) {$1$};
    \node[list,on chain=xs] (B) {$2$};
    \node[list,on chain=xs] (C) {$3$};
    \node[list,on chain=xs] (D) {$4$ \nodepart{two} $\bullet$};

    \draw[->] let \p1 = (E.east), \p2 = (E.center) in (\x1,\y2) -- (A);
    \draw[->] let \p1 = (A.two), \p2 = (A.center) in (\x1,\y2) -- (B);
    \draw[->] let \p1 = (B.two), \p2 = (B.center) in (\x1,\y2) -- (C);
    \draw[->] let \p1 = (C.two), \p2 = (C.center) in (\x1,\y2) -- (D);

  \end{tikzpicture}
  \vspace{\baselineskip}

  \begin{tikzpicture}[
    list/.style={
        rectangle split,
        rectangle split parts=2,
        draw,
        rectangle split horizontal
      },
      >=stealth,
      start chain=xs,
      node distance=1em
    ]

    \node (E) {output};
    \node[right=of E, on chain=xs] (p1) {p1};
    \node[right=of E, list,on chain=xs] (A) {$1$};
    \node[list,on chain=xs] (B) {$2$};
    \node[list,on chain=xs] (C) {$3$};
    \node[list,on chain=xs] (D) {$4$ \nodepart{two} $\bullet$};

    \node[above=of B] (p2) {p2};
    \node[above=of C] (p3) {p3};
    \node[above=of D] (p4) {p4};

    \draw[double distance=0.05em, ->]
      let \p1 = (E.east), \p2 = (E.center) in (\x1,\y2) -- (p1);
    \draw[->] let \p1 = (A.two), \p2 = (A.center) in (\x1,\y2) -- (B);
    \draw[->] let \p1 = (B.two), \p2 = (B.center) in (\x1,\y2) -- (C);
    \draw[->] let \p1 = (C.two), \p2 = (C.center) in (\x1,\y2) -- (D);

    \draw[->] let \p1 = (p1.east), \p2 = (A) in (\x1,\y2) -- (A);
    \draw[->] let \p1 = (p2), \p2 = (p4.south) in (\x1,\y2) -- (B);
    \draw[->] let \p1 = (p3), \p2 = (p4.south) in (\x1,\y2) -- (C);
    \draw[->] let \p1 = (p4), \p2 = (p4.south) in (\x1,\y2) -- (D);

  \end{tikzpicture}
\end{figure*}

\section{Exercise 2}

We keep track of the last element for which $<$ returned false in a \mintinline[bgcolor=LightGray]{haskell}{Maybe a}.
\inputminted
  [bgcolor=LightGray,firstline=27, lastline=32]
  {haskell}
  {src/UnbalancedSet.hs}
Then we must modify the typeclass method to use it.
\inputminted
  [bgcolor=LightGray,firstline=12, lastline=17]
  {haskell}
  {src/UnbalancedSet.hs}

Here and unless stated otherwise, we refer to the Haskell source in Appendix A of the text.

\section{Exercise 3}

I'm not sure Haskell has the same notion of handlers or exceptions. In Haskell we don't worry about the copying due to garbage collection.

\section{Exercise 4}

\end{document}
